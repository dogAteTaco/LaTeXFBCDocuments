\subsection*{Kyiv (AWE-37)}
\subsubsection*{DETAILS:}
\par Recordings of the audio phenomena were uploaded onto the
internet shortly after the event. These records circulated rapidly
on popular message boards. The Communications Department
utilized this exposure by creating "Sky Trumpet hoax" videos and
posting related theories to spread confusion and draw attention
away from the event's paranatural origin. Industrial noise,
particularly the sound of metal drilling was found to be a widely-
accepted explanation. Theories about the sounds emanating from
the Earth itself, known as Seismic Hum, emerged from the public
itself and were encouraged by the Bureau to generate further
misdirection and eventual public disinterest.
\par Witnesses of the event were monitored discretely afterwards.
Observed symptoms were consistent with \censor{very longest t}
deprivation, but subsided after 12-15 days. The length of the
symptoms directly correlated to the individual's proximity
(unsheltered) to the supposed epicenter. One linked, although
accidental, casualty can be listed (see report re: the effects of
planar friction on hearing aids in file 14-200-1010).