\subsection*{Arctic Queen (AI10-KE)}
\subsubsection*{CONTAINMENT PROCEDURE:}
\par No unique procedures required.
\subsubsection*{DESCRIPTION/ALTERED EFFECT:}
\par An "Arctic Queen" brand electric
refrigerator model from the 1960s. No
cooling functionality. The door is
decorated with crayon illustrations by \censor{very longest text right}. The
illustrations cannot be removed from the item. All attempts have
failed. The paper cannot be burned or torn.
\subsubsection*{BACKGROUND:}
\par The item first came to the Bureau's attention after it survived the
collapse of New York City's Grand Central Hotel, where it served
as an appliance in apartment \censor{longer text}, rented by a man named \censor{super mega longer text}. It became the subject of local infamy after surviving the building's collapse undamaged. Mr. \censor{Johansson}, who was out of the building at the time of collapse (\censor{seventeen}p.m.), retained ownership of the item until 1974, when
the Bureau purchased the item through a false identity.
\subsubsection*{APPENDIX:}
\par New altered effect has been observed. See file All O.F for details.
See file All O.K for new containment procedures.